\chapter{Introduction\label{cha:introduction}}
%% \ifdraft only shows the text in the first argument if you are in draft mode.
%% These directions will disappear in other modes.
State the objectives of the exercise. Ask yourself:
\underline{Why} did I design/create the item? What did I aim to
achieve? What is the problem I am trying to solve?  How is my
solution interesting or novel?

\section{Background}
Provide background about the subject matter (e.g. How was morse code
developed?  How is it used today?.

This is a place where there are usually many citations.
It is suspicious when there is not.
Include the purpose of the different equipment and your design intent. 
Include references to relevant scientific/technical work and books.
What other examples of similar designs exist?
How is your approach distinctive?

If you have specifications or related standards, these must be
described and cited also.  As an example, you might cite the specific
RoboSub competition website (and documents) if working on the lighting system for an AUV\cite{guls2016auvlight}\index{AUV}

\section{Example Section}
\begin{figure}
  \centering
  \includegraphics[width=0.8\textwidth]{design-commonalities}
  \caption[Design]{Design Commonalities\cite{foley2021dindesign}}\label{fig:ru-logo}
\end{figure}
\begin{table}
  \centering
  \begin{tabular}{ll}\toprule
    $x$& $x^{2}$\\\midrule
    1 &1\\
    2 &4\\
    3 &9\\\bottomrule
  \end{tabular}
  \caption{Table of squared numbers}\label{tab:numbers}
\end{table}
There is an example of how to map design methods in CDIO, Axiomatic Design, and Product Design in Figure~\ref{fig:ru-logo}.
This image will scale according to the width of the text on the page.
There is a helpful list of squared numbers in Table~\ref{tab:numbers}.
This table is formatted in the style of a book, which is very differerent than the style one is used to in Excel.

The test text ``Lorem Ipsum''\index{Lorem Ipsum} is from an ancient text from 45 B.C. \cite{cicero46deFinibus, lipsomwebsite}\\
\lipsum[1-5]
\subsection{Subsection}
\lipsum[6-10]
\subsubsection{SubSubsection} 
\lipsum[11-15]
\section[Section with an extremely long name]{Section with a very very very very very very very very very very very very very very very very very very very very long name}
\lipsum[11-18]

%%% Local Variables:
%%% mode: latex
%%% TeX-master: "main"
%%% TeX-engine: luatex
%%% End:
