%%%% This is the general thesis/project report template for most users
\documentclass[showtrims]{rubook}
%% RU Book options (subclass of memoir)
%% a4paper or b5paper(default):  paper stock size.
%%     If A4, show cut lines.  If b5, no cut lines.
\usepackage[]{ruthesis}
%% Options for ruthesis in []:
%%   IS Icelandic is main language, otherwise default to English

%\usepackage{apacite}
%% Uncomment the apacite package for APA style citations
%% You will also need to comment out the line (around line 134)
%% % \bibliographystyle{IEEEtran/bibtex/ieeetran}

%%%%%% Packages and Macros %%%%%%%%%%%%%%%%%%%%%%%%%%%%%%%%
\usepackage{custom}
%% Commonly-used packages and macros are in custom.sty
%% Put any additional packages after this line
%% !!WARNING: The geometry package is incompatible with this template!

\usepackage{lipsum}%provides us with text for testing
%% usage: \lipsum[STARTNUM-ENDNUM]

\graphicspath{{graphics/}{Graphics/}{./}}
%% This is a list of folders to search for graphics files to include
%% for the graphicx package (already loaded).  This may be case-sensitive.
%% LaTeX will search from left to right in the list, so you can put "cropped" versions
%% in the first directory and it will grab them first. e.g.
%\graphicspath{{graphics-cropped/}{graphics/}{Graphics/}{./}}

%%%%%%%%%%%%%%%%%%%%%%%%%%%%%%%%%%%%%%%%%%%%%%%%%%%%%%%

\title{Thesis and Project Report Template for \theInstitution{}}
\author{Joseph~T.~Foley and Marcel~Kyas}
\date{2022}{5}{23}%{Year}{Month}{Day}%use numbers

%\DocumentInfo{TYPE}{ABBREVIATION}{DEGREE}{PROGRAM}{ECTS}{School/Department}
\DocumentInfo{Thesis}{M.Sc.}{Master of Science}{Mechatronics}{30}{Department of Engineering}
\DocumentDescription{\theDocumentType{} of \theECTS{} ECTS credits
         submitted to the \theSchool{} at \theInstitution{} in partial fulfillment
         of the requirements for the degree of \theDegreeLong}
% Change this if you need a custom title or if it needs to be in Icelandic

%% PhD only have Thesis Committee with roles.  Examiner is part of committee.
\SupervisorHeading{Thesis Committee}
\Supervisors{
  \personinfo{Superior A. Teacher}{Supervisor}{Professor}{Reykjavik University}{Iceland}
  \personinfo{Helpful A. Teacher}{Co-advisor}{Assistant Professor}{University of Iceland}{Iceland}
  \personinfo{Tough E. Questions}{Examiner}{Associate Professor}{Massachusetts Institute of Technology}{USA}
}

\begin{document}
%% TODO: get the official cover graphic and have the system fill in the fields for you
\maketitle{}
\copyrightpage{RU Report Template}{0000-0000-0000-0000}{100}{ISBN 978-XXXX-XXXX-X-X (print version)\\
ISBN 978-XXXX-XXXX-X-X (electronic version)}{Printing: RU Printing osf.}{Printed on 900g Cardboard paper}
% If this is a PhD, register for an ISSN and ISBN, then:
% \copyrightpage{<Short title>}{<ORCID>}{<total number of pages>}{ISBN 978-XXXX-XXXX-X-X (print version)\\
% ISBN 978-XXXX-XXXX-X-X (electronic version)}{Printing: <Name of printer>}{Printed on <type of paper>}
% For submitting in Skemman the ISBN can be replaced with the url handle ( \url{http://hdl.handle.net/1946/xxxx} )
% Inside ruthesis.sty, the copyrightpage can be edited - there is an option to include a stamp/icon of the printer

%\signaturepage{} %Generally only for Print copies & now defunct in the Engineering department.


%\begin{dedications}% Optional
%  I dedicate this to my spouse/child/pet/power animal.
%\end{dedications}

\enableindents{}% turn on/off paragraph indents
% RUM: "Acknowledgements (optional)"%start numbering

\clearpage{}
\tableofcontents{}\clearpage
\listoffigures{}\clearpage
\listoftables{}\clearpage

%% A list of abbreviations is an example of a special list
%% Other lists may be added, such as lists of algorithms, symbols, theorems, etc.
%% IN CS PhD, this is sometimes centered.
\chapter*{List of Symbols}%%RUM: Not mentioned
%% This list demonstrates the "siunitx" package capabilities
\begin{tabular}{lll}
Symbol &Description &Value/Units\\
$E$ &Energy &\si{\joule}\\ %New function:  \unit{} in Livetex 2021
$m$ &Mass &\si{\gram}\\ %New function:  \unit{} in Livetex 2021
$c$ &Speed of Light &\SI{2.99E9}{\meter\per\square\second}\\ %New function:  \qty{} in Livetex 2021
\end{tabular}

\begin{abstract}
  The goal of this template is to produce electronic output to be uploaded to Skemman that can be later printed out and bound into a professional looking textbook that fits on standard library shelves.
  It is important to note that A4 paper when bound requires taller shelf spacing, so the B5 paper format was chosen instead.
  When binding a book, the edges that face outward need to be very smooth to reduce contamination and dust from entering the book when it sits on a shelf; this is why traditionally a larger paper size is cut down to the book size.
  If your print house expects the stock to be A4, then make sure the rubook has the ``a4paper'' option.
  If they prefer to deal with preparation themselves from a B5 pdf, then the default ``b5paper'' option is correct.
  The template is optimized for lualatex, but should still work with pdflatex.
  
  The abstract goes here in English or Icelandic.
  It should be a fairly short summary of the entire document.
  If you switch to Icelandic mode (IS option to ruthesis) then abstract will become \'{U}tdr\'{a}ttur

  Keywords / Efnisord:  Keywords, separated, by, commas
\end{abstract}


% RUM: "Acknowledgements (optional)"
\chapter*{Acknowledgments} 
\begin{quotation}
So long, and thanks for all the fish.
\end{quotation}\sourceatright{Douglas Adams\cite{adams84fish}}
\vspace{\baselineskip}

This work was funded by \the\year~RANNIS grant ``Survey of man-eating Minke whales'' 1415550.
Additional equipment was generously donated by the Icelandic Tourism Board.

{\em Acknowledgements are optional; comment this chapter out if they are absent
  Note that it is important to acknowledge any funding that helped in the work\/}
%\/ is italic correction at the end of \em and italics
\clearpage{}

\mainmatter{}
\aliaspagestyle{chapter}{empty}
% Don't put page numbers on the chapter changes

%% If you would like to separate chapters into different files, use
%% \include{chapterfile}
%% WARNING: Make sure that all of these files (and any new ones)
%% are UTF-8 otherwise you will get weird encoding errors.
\part{Getting Started} % Parts optional but useful in longer documents
\chapter{Instructions}
\textbf{Is this the right template for you?}
If you are not a Reykjavik University student, this is probably not for you.
For everyone else, please read the instructions before getting started.
It is likely to save you a lot of frustration and errors.
\input{instructions}%\input{} just adds the code
\part{Demonstration}
\include{introduction}%include{} starts a new page
\section{Another Section}
\part{The Second Part} % Parts optional but useful in longer documents

\bibliographystyle{IEEEtran/bibtex/ieeetran}
\bibliography{references}

%% If appendices are needed, uncomment the following line
%% and include the appendices in separate files
\appendix{}%%RUM: "Appendicies (as appropriate)
\input{code} % as an example, perhaps some of your code

%\backmatter{} % Sections after this don't get numbers
%% We prefer that all elements be numbered

%%%%%%%%%%%%% SHOW INDEX %%%%%%%%%%%%%%%%%%
%% Index, optional.  A good idea on longer documents
\clearforchapter{}
\printindex{}%%RUM: Not mentioned

\end{document}
%%% Local Variables:
%%% mode: latex
%%% TeX-master: t
%%% TeX-engine: luatex
%%% End:
