%%%% WARNING:  This is the general template for most users
%%%% if you are writing a PhD or SciD dissertation
%%%% please use the dissertation.tex file instead
\documentclass[12pt,a4paper,titlepage]{memoir}
\usepackage[]{ruthesis}
%% Options in []:
%% IS Icelandic is main language
\thesistitle{Thesis and Project Report Template for \theInstitution{}.}{Thesis and Project Report Template for \theInstitution{} in Icelandic}
\author{Joseph T. Foley}%Use \and as an author separator
\date{2022}{2}{2}%{Year}{Month}{Day}%use numbers

%\DocumentInfo{TYPE}{ABBREVIATION}{DEGREE}{PROGRAM}
\DocumentInfo{Thesis}{M.Sc.}{Master of Science}{Mechatronics}
% TODO:  department?
\School{School of Technology}

%% PhD only have Thesis Committee with roles.  Examiner is part of committee.
\SupervisorHeading{Thesis Committee}
\Supervisors{
  \personinfo{Superior A. Teacher}{Supervisor}{Professor}{Reykjavik University}{Iceland}
  \personinfo{Helpful A. Teacher}{Co-advisor}{Assistant Professor}{University of Iceland}{Iceland}
  \personinfo{Tough E. Questions}{Examiner}{Associate Professor}{Massachusetts Institute of Technology}{USA}
}

%%%%%% Useful Packages %%%%%%%%%%%%%%%%%%%%%%%%%%%%%%%%
\usepackage{lipsum}%provides us with text for testing
%% usage: \lipsum[STARTNUM-ENDNUM]

\usepackage[final]{listings}
%%% Formatting code inclusion and snippets
%% "final" option to force it to display code

\usepackage{siunitx}
%% \SI{9.82}{\meter\per\second\square}
%%%%%%%%%%%%%%%%%%%%%%%%%%%%%%%%%%%%%%%%%%%%%%%%%%%%%%%

\begin{document}

\maketitle{}
\copyrightpage{}
\signaturepage{}
\archivesigpage{}

\begin{abstract}
  The abstract goes here translated into English.
  If the thesis is in English, it should come first.
  It should be a fairly short summary of the entire document.
\end{abstract}

\begin{abstractIS}
  The abstract goes here translated into Icelandic.
  If the thesis is in Icelandic, it should come first.
  It should be a fairly short summary of the entire document.
\end{abstractIS}

\begin{dedications}
  I dedicate this to my spouse/child/pet/power animal.
\end{dedications}

\enableindents{}% turn on/off paragraph indents
% RUM: "Acknowledgements (optional)"%start numbering

\chapter*{Acknowledgements} 
\begin{quotation}
So long, and thanks for all the fish.
\end{quotation}\sourceatright{Douglas Adams\cite{adams84fish}}
\vspace{\baselineskip}

This work was funded by \the\year~RANNIS grant ``Survey of man-eating Minke whales'' 1415550.
Additional equipment was generously donated by the Icelandic Tourism Board.

{\em Acknowledgements are optional; comment this chapter out if they are absent
  Note that it is important to acknowledge any funding that helped in the work}

\clearpage{}
\tableofcontents{}\clearpage
\listoffigures{}\clearpage
\listoftables{}\clearpage

%% The list of abbreviations is an example of a special list
%% Other lists may be added, such as lists of algorithms, symbols, theorems, etc.
%% IN CS PhD, this is sometimes centered.
\chapter*{List of Abbreviations}%%RUM: Not mentioned
\begin{tabular}{ll}
M.Sc. &Masters of Science\\
Ph.D. &Doctor of Philosophy\\
\end{tabular}

\chapter*{List of Symbols}%%RUM: Not mentioned
\begin{tabular}{lll}
Symbol &Description &Value/Units\\
$E$ &Energy &\si{\joule}\\
$m$ &Mass &\si{gram}\\
$c$ &Speed of Light &\SI{2.99E8}{\meter\per\second\square}\\
\end{tabular}


\mainmatter{}
\aliaspagestyle{chapter}{empty}
% Don't put page numbers on the chapter changes

%% If you would like to separate chapters into different files, use
%% \include{chapterfile}
%% WARNING: Make sure that all of these files (and any new ones)
%% are UTF-8 otherwise you will get weird encoding errors.
\chapter{Working with the template}
\section{\LaTeX{} Hints}
\begin{itemize}
\item Put one sentence per line.
  This makes it easier to debug errors (which are by line) and to do grammar checking with \url{http://grammarly.com}.
\item Compile the document often and look for errors.
  If you find one, try commenting out the area to locate the source of the problem.
\item Watch out for \& and \%.  They have to have a left-slash in front of them.
\end{itemize}

\chapter{Common Document Structure with Examples}
\section{Introduction}
\jfnote{Don't forget to write the introduction.}
\jfwarning{Really, don't forget to write the introduction.}
What is the idea?  What is it called and why?
Who is the target customer?

\subsection{Customer Needs}
What would a customer need the item to do?  
Using Axiomatic Design theory, this is stated as a numbered list of Customer Needs(CN)~\cite{suh1990principles}.
The top level is \CN0.
This is often (but not always) decomposed into \CN1, \CN2, etc.
Here is an example of a top level:

\begin{quote} \textbf{\CN0} A transfer bin for whole salmon, compatible with the SureTrack grader, cheaper and less prone to cracking due to skewing.  
The bin should be adaptable to a pure transfer task and be able to discharge anywhere along its path without
accidental discharge.~\cite{gerhard2016suretrack}
\end{quote}


\section{Prior Art}
What exists that is similar?  How is yours better/distinctive?
Give at least two examples and quantify the differences (numeric values).
If you say something is cheaper, you need to give the costs for both items.

An example of a figure is the grey square in Fig.~\ref{fig:grey-square}.
\begin{figure}
  \centering
  \includegraphics[width=\columnwidth]{grey-square}
  \caption{Grey square.}\label{fig:grey-square}
\end{figure}



\subsection*{Sources}
You will want to cite all these similar concepts/products.
As an example of a citation, Carryer et al.~\cite{carryer2011IntroMechatronics} is the textbook for T-411-MECH Mechatronics 1.


\section{Design}
As previously mentioned, using Axiomatic Design Theory is a good way to develop your design.

Here is a brief synopsis from Omarsdóttir et al.\cite{omarsdottir2016chessmate}:
\begin{quotation}
  Rather, the focus was placed on developing comprehensive FR and DP lists, then evaluating the coupling between them.
  This coupling is symbolized in a design matrix, which is a Cartesian product of all FR and DP combinations~\cite{cochran2016msdd, benevides2012aed}.
Where there is an interaction between an FR and DP, this is denoted by a non-zero coefficient, or in the case of the value being unknown, simply a placeholder variable $X$.
Minor levels of coupling, often considered higher-order effects, are annotated with $x$ to show their lessened effect.
A diagonal matrix is ``uncoupled'' and satisfies the Independence Axiom: ``to maintain the independence of the functional requirements~(FRs)''~\cite{suh2001axiomatic}.
Such a design can be easily optimized by adjusting a particular FR or DPs without affecting others.
A diagonal matrix indicates a ``decoupled'' or ``path-dependent'' solution, which can still be optimized, but the ordering of parameter choice selection becomes important.
All other design matrices are ``coupled'' and may have a usable local solution but usually resist modification and optimization~\cite{suh2001axiomatic}.
Needless to say, the focus is on minimizing coupling wherever it may appear.

ADT's second axiom is ``minimize the information content of the design.''
Simply put, ensure that the design has the highest probability of meeting the stated FRs.
When systems are not able to meet FRs all of the time, this is denoted in ADT as ``complexity'' and is deeply explored in~\cite{suh2005complexity}.
As will become apparent in the next section, this axiom became integral to the design of the interaction between the robot and its chess pieces.
Finally, any factors to be considered that are not functional are categorized as ``Constraints.''
These are often resource-focused and affect all of the design decisions; they need to be revisited often especially when choosing between otherwise equivalent implementations.
\end{quotation}
The first axiom is often called the Independence Axiom, and the second, the Information Axiom.


From the Customer Needs, we build a list of Functional Requirements.

Again, we start with a top-level \FR0: ``Contain \SI{25}{\kilogram} of fish on SureTrack conveyor until release is triggered''
From this, a top-level Design Parameter \DP0: Gable-reinforced stainless-steel locking bin with bi-directional discharge
\cite{gerhard2016suretrack}.

We continue a ``zig-zag'' procedure to decompose and map the FRs to the DPs as shown in Table~\ref{tab:first_level-frdp}.

\begin{table}
  \center
  \caption{First level FR-DP mapping.~\cite{gerhard2016suretrack}}\label{tab:first_level-frdp}
  \begin{tabular}{lll} \toprule
    ID& Functional Requirement & Design Parameter \\ \midrule 
    1&Contain product&Main weldment\\
    2&Move product&Support system\\
    3&Discharge product &Discharge system\\
    \bottomrule
  \end{tabular}
\end{table}

From this mapping we develop a design matrix as shown in Equation~\ref{eq:top-design-matrix} from~\cite{gerhard2016suretrack}.

\begin{equation}\label{eq:top-design-matrix}
\begin{Bmatrix}
\FR{1}\\
\FR{2}\\
\FR{3}
\end{Bmatrix}=
\begin{bmatrix}
X &  0 & X\\
0 &  X & 0\\
0 &  0 & X
\end{bmatrix}
\begin{Bmatrix}
\DP{1}\\
\DP{2}\\
\DP{3}
\end{Bmatrix}
\end{equation}

This matrix is de-coupled i.e.\ path-dependent, meaning it can be optimized, but the order matters.

\section{Experiments}

\section{Results and Discussion}

\section{Conclusion}

\subsection{Future work}

\subsection{Summary}

\part{The First Part} % Parts optional but useful in longer documents
\chapter*{Is this the right template}
If you are not a Reykjavik University student, this is probably not for you.
If you are writing a Ph.D. or Sci.D. dissertation, then you want the file \path{dissertation.tex}.
\chapter{The First Chapter}
\include{introduction}%Chapter in introduction.tex
\section{Another Section}
\part{The Second Part} % Parts optional but useful in longer documents

\bibliographystyle{IEEEtran/bibtex/ieeetran}
\bibliography{references,references-collections}
%% If you use bibtex "hyperref", you will need to make sure that the Proceedings, Book, or other things is included last.
%% The author puts his in references-collections.bib to ensure that they are always last.

%% If appendices are needed, uncomment the following line
%% and include the appendices in separate files
\appendix{}%%RUM: "Appendicies (as appropriate)
\input{code} % as an example, perhaps some of your code

%\backmatter{} % Sections after this don't get numbers
%% We prefer that all elements be numbered

%%%%%%%%%%%%% SHOW INDEX %%%%%%%%%%%%%%%%%%
%% Index, optional.  A good idea on longer documents
\clearforchapter{}
\printindex{}%%RUM: Not mentioned

\end{document}
%%% Local Variables:
%%% mode: latex
%%% TeX-master: t
%%% End:
